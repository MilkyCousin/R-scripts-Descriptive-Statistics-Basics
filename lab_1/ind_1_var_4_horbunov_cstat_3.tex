%% ================================================================================
%% This LaTeX file was created by AbiWord.                                         
%% AbiWord is a free, Open Source word processor.                                  
%% More information about AbiWord is available at http://www.abisource.com/        
%% ================================================================================

\documentclass[a4paper,portrait,12pt]{article}
\usepackage[latin1]{inputenc}
\usepackage{calc}
\usepackage{setspace}
\usepackage{fixltx2e}
\usepackage{graphicx}
\usepackage{multicol}
\usepackage[normalem]{ulem}
\usepackage{color}
\usepackage{hyperref}
\usepackage[english,russian]{babel}

\begin{document}


\begin{flushleft}
Самостiйна робота з дескриптивної статистики
\end{flushleft}


\begin{flushleft}
Горбунов Данiел Денисович,
\end{flushleft}


\begin{flushleft}
III курс бакалаврату,
\end{flushleft}


\begin{flushleft}
комп'ютерна статистика
\end{flushleft}


\begin{flushleft}
Варiант №4
\end{flushleft}


\begin{flushleft}
8 вересня 2019 р.
\end{flushleft}





1





\begin{flushleft}
Завдання №1
\end{flushleft}





1.1





\begin{flushleft}
Данi
\end{flushleft}





\begin{flushleft}
Для мiст України: Київ, Днiпро, Одеса, Львiв, Харкiв, Iвано-Франкiвськ, Житомир:
\end{flushleft}


\begin{flushleft}
1. Перший набiр: цiни на абонемент на 1 мiсяць у фiтнес-клубi на одну особу;
\end{flushleft}


\begin{flushleft}
2. Другий набiр: цiни на 1 кiлограм помiдорiв.
\end{flushleft}


\begin{flushleft}
Для кожного набору обчислiть такi дескриптивнi статистики вибiрки:
\end{flushleft}


\begin{flushleft}
$\bullet$ Вибiркове середнє;
\end{flushleft}


\begin{flushleft}
$\bullet$ Середнє геометричне;
\end{flushleft}


\begin{flushleft}
$\bullet$ Середнє гармонiйне;
\end{flushleft}


\begin{flushleft}
$\bullet$ Медiана;
\end{flushleft}


\begin{flushleft}
$\bullet$ Середина дiапазону;
\end{flushleft}


\begin{flushleft}
$\bullet$ Дисперсiя;
\end{flushleft}


\begin{flushleft}
$\bullet$ Середньоквадратичне вiдхилення;
\end{flushleft}


\begin{flushleft}
$\bullet$ Iнтерквартильний розмах;
\end{flushleft}


\begin{flushleft}
$\bullet$ Ширина дiапазону;
\end{flushleft}


\begin{flushleft}
$\bullet$ Коефiцiєнт варiацiї.
\end{flushleft}





1





\begin{flushleft}
\newpage
Данi з файлiв countries\_fitness.csv та countries\_tomatoes.csv.
\end{flushleft}


\begin{flushleft}
Names
\end{flushleft}


\begin{flushleft}
Values
\end{flushleft}


\begin{flushleft}
Kiev
\end{flushleft}


526.96


\begin{flushleft}
Dnipro
\end{flushleft}


388.81


\begin{flushleft}
Odesa
\end{flushleft}


559.75


\begin{flushleft}
Lviv
\end{flushleft}


458.16


\begin{flushleft}
Kharkiv
\end{flushleft}


489.5


\begin{flushleft}
Ivano-Frankivsk
\end{flushleft}


420


\begin{flushleft}
Zhytomyr
\end{flushleft}


351





1.2





\begin{flushleft}
Names
\end{flushleft}


\begin{flushleft}
Values
\end{flushleft}


\begin{flushleft}
Kiev
\end{flushleft}


39.78


\begin{flushleft}
Dnipro
\end{flushleft}


38.08


\begin{flushleft}
Odesa
\end{flushleft}


35.89


\begin{flushleft}
Lviv
\end{flushleft}


34.1


\begin{flushleft}
Kharkiv
\end{flushleft}


24.33


\begin{flushleft}
Ivano-Frankivsk
\end{flushleft}


32.5


\begin{flushleft}
Zhytomyr
\end{flushleft}


27.05





\begin{flushleft}
Формули обчислення та їх реалiзацiя в R
\end{flushleft}





\begin{flushleft}
Нехай X = \{X1 , ..., Xn \} - певна вибiрка, де:
\end{flushleft}


\begin{flushleft}
Xj - значення дослiджуваної змiнної у j-тому спостереженнi,
\end{flushleft}


\begin{flushleft}
n - кiлькiсть елементiв у вибiрцi.
\end{flushleft}


\begin{flushleft}
Варiацiйний ряд має наступний вигляд: min Xj = X[1] 6 X[2] 6 ... 6 X[n] = max Xj
\end{flushleft}


\begin{flushleft}
16j6n
\end{flushleft}





\begin{flushleft}
16j6n
\end{flushleft}





\begin{flushleft}
Вибiркове середнє для вибiрки X визначається за формулою:
\end{flushleft}


\begin{flushleft}
n
\end{flushleft}





\begin{flushleft}
1X
\end{flushleft}


\begin{flushleft}
X=
\end{flushleft}


\begin{flushleft}
Xj
\end{flushleft}


\begin{flushleft}
n j=1
\end{flushleft}





(1)





\begin{flushleft}
Середнє геометричне, що визначається для таких вибiрок X, у яких значення змiнної
\end{flushleft}


\begin{flushleft}
Xj приймають лише додатнi значення:
\end{flushleft}


\begin{flushleft}
v
\end{flushleft}


\begin{flushleft}
uY
\end{flushleft}


\begin{flushleft}
n
\end{flushleft}


\begin{flushleft}
u
\end{flushleft}


\begin{flushleft}
n
\end{flushleft}


\begin{flushleft}
GM (X) = t
\end{flushleft}


\begin{flushleft}
Xj
\end{flushleft}


(2)


\begin{flushleft}
j=1
\end{flushleft}





\begin{flushleft}
Середнє гармонiйне дорiвнює наступному:
\end{flushleft}


\begin{flushleft}
HM (X) = Pn
\end{flushleft}





\begin{flushleft}
n
\end{flushleft}





1


\begin{flushleft}
j=1 Xj
\end{flushleft}





\begin{flushleft}
Вибiркова медiана для вибiрки X обчислюється за формулою:
\end{flushleft}


(


\begin{flushleft}
X[(n+1)/2] ,
\end{flushleft}


\begin{flushleft}
якщо n - непарне;
\end{flushleft}


\begin{flushleft}
med(X) = 1
\end{flushleft}


\begin{flushleft}
(X[n/2] + X[n/2+1] ), якщо n - парне.
\end{flushleft}


2





(3)





(4)





\begin{flushleft}
Середина дiапазону:
\end{flushleft}


1


(5)


\begin{flushleft}
M R(X) = (X[1] + X[n] )
\end{flushleft}


2


\begin{flushleft}
Будемо застосовувати формулу для обчислення виправленої вибiркової дисперсiї.
\end{flushleft}


\begin{flushleft}
Вiдрiзняється вiд звичайної нормуючим множником (n $-$ 1)/n:
\end{flushleft}


\begin{flushleft}
n
\end{flushleft}





\begin{flushleft}
S02 (X)
\end{flushleft}





\begin{flushleft}
1 X
\end{flushleft}


=


\begin{flushleft}
(Xj $-$ X)2
\end{flushleft}


\begin{flushleft}
n $-$ 1 j=1
\end{flushleft}


2





(6)





\begin{flushleft}
\newpage
Середньоквадратичне вiдхилення - квадратний корiнь вiд значення вибiркової дисперсiї:
\end{flushleft}


\begin{flushleft}
v
\end{flushleft}


\begin{flushleft}
u
\end{flushleft}


\begin{flushleft}
n
\end{flushleft}


\begin{flushleft}
q
\end{flushleft}


\begin{flushleft}
u 1 X
\end{flushleft}


2


\begin{flushleft}
(Xj $-$ X)2
\end{flushleft}


(7)


\begin{flushleft}
S0 (X) = S0 (X) = t
\end{flushleft}


\begin{flushleft}
n $-$ 1 j=1
\end{flushleft}


\begin{flushleft}
Iнтерквартильний розмах:
\end{flushleft}


\begin{flushleft}
IQ(X) = Q3 (X) $-$ Q1 (X),
\end{flushleft}





(8)





\begin{flushleft}
де Q3 (X) та Q1 (X) визначаються наступним чином:
\end{flushleft}


\begin{flushleft}
Q2 (X) = med(X)
\end{flushleft}


\begin{flushleft}
Q1 (X) = med(\{X[j] $\in$ X| min Xj = X[1] 6 X[2] 6 ... 6 X[n] = Q2 (X)\})
\end{flushleft}


\begin{flushleft}
16j6n
\end{flushleft}





\begin{flushleft}
Q3 (X) = med(\{X[j] $\in$ X|Q2 (X) = X[1] 6 X[2] 6 ... 6 X[n] = max Xj \})
\end{flushleft}


\begin{flushleft}
16j6n
\end{flushleft}





\begin{flushleft}
Ширина дiапазону:
\end{flushleft}


\begin{flushleft}
Range(X) = X[n] $-$ X[1]
\end{flushleft}





(9)





\begin{flushleft}
Коефiцiєнт варiацiї:
\end{flushleft}


\begin{flushleft}
CV (X) =
\end{flushleft}





3





\begin{flushleft}
S0 (X)
\end{flushleft}


\begin{flushleft}
X
\end{flushleft}





(10)





\begin{flushleft}
\newpage
Маючи всi необхiднi формули для обчислення статистик, спробуємо реалiзувати кожну в
\end{flushleft}


\begin{flushleft}
R.
\end{flushleft}


\begin{flushleft}
1. Вибiркове середнє:
\end{flushleft}


\begin{flushleft}
sample\_mean $<$- f u n c t i o n ( sample )
\end{flushleft}


\{


\begin{flushleft}
r e s u l t $<$- mean ( sample )
\end{flushleft}


\begin{flushleft}
result
\end{flushleft}


\}





\begin{flushleft}
2. Середнє геометричне:
\end{flushleft}


\begin{flushleft}
g e o m e t r i c\_mean $<$- f u n c t i o n ( sample )
\end{flushleft}


\{


\begin{flushleft}
r e s u l t $<$- prod ( sample ) \^{} ( 1 / l e n g t h ( sample ) )
\end{flushleft}


\begin{flushleft}
result
\end{flushleft}


\}





\begin{flushleft}
3. Середнє гармонiйне:
\end{flushleft}


\begin{flushleft}
harmonic\_mean $<$- f u n c t i o n ( sample )
\end{flushleft}


\{


\begin{flushleft}
r e s u l t $<$- l e n g t h ( sample ) /sum ( 1 / sample )
\end{flushleft}


\begin{flushleft}
result
\end{flushleft}


\}





\begin{flushleft}
4. Вибiркова медiана:
\end{flushleft}


\begin{flushleft}
sample\_median $<$- f u n c t i o n ( s o r t e d\_sample )
\end{flushleft}


\{


\begin{flushleft}
r e s u l t $<$- median ( s o r t e d\_sample )
\end{flushleft}


\begin{flushleft}
result
\end{flushleft}


\}





\begin{flushleft}
5. Середина дiапазону:
\end{flushleft}


\begin{flushleft}
mid\_r a n g e $<$- f u n c t i o n ( sample )
\end{flushleft}


\{


\begin{flushleft}
r e s u l t $<$- 0 . 5 * (max( sample ) + min ( sample ) )
\end{flushleft}


\}





4





\begin{flushleft}
\newpage
6. Вибiркова дисперсiя:
\end{flushleft}


\begin{flushleft}
v a r i a n c e $<$- f u n c t i o n ( sample )
\end{flushleft}


\{


\begin{flushleft}
r e s u l t $<$- var ( sample )
\end{flushleft}


\begin{flushleft}
result
\end{flushleft}


\}





\begin{flushleft}
7. Середньоквадратичне вiдхилення:
\end{flushleft}


\begin{flushleft}
s t a n d a r d\_d e v i a t i o n $<$- f u n c t i o n ( sample )
\end{flushleft}


\{


\begin{flushleft}
r e s u l t $<$- sd ( sample )
\end{flushleft}


\begin{flushleft}
result
\end{flushleft}


\}





\begin{flushleft}
8. Iнтерквартильний розмах:
\end{flushleft}


\begin{flushleft}
i q $<$- f u n c t i o n ( s o r t e d\_sample )
\end{flushleft}


\{


\begin{flushleft}
q2 $<$- sample\_median ( s o r t e d\_sample )
\end{flushleft}


\begin{flushleft}
q1 $<$- sample\_median ( s o r t e d\_sample [ s o r t e d\_sample $<$= q2 ] )
\end{flushleft}


\begin{flushleft}
q3 $<$- sample\_median ( s o r t e d\_sample [ s o r t e d\_sample $>$= q2 ] )
\end{flushleft}


\begin{flushleft}
r e s u l t $<$- q3 - q1
\end{flushleft}


\begin{flushleft}
result
\end{flushleft}


\}





\begin{flushleft}
9. Ширина дiапазону:
\end{flushleft}


\begin{flushleft}
sample\_r a n g e $<$- f u n c t i o n ( sample )
\end{flushleft}


\{


\begin{flushleft}
r e s u l t $<$- max( sample ) - min ( sample )
\end{flushleft}


\begin{flushleft}
result
\end{flushleft}


\}





\begin{flushleft}
10. Коефiцiєнт варiацiї:
\end{flushleft}


\begin{flushleft}
cv $<$- f u n c t i o n ( sample )
\end{flushleft}


\{


\begin{flushleft}
r e s u l t $<$- s t a n d a r d\_d e v i a t i o n ( sample ) / sample\_mean ( sample )
\end{flushleft}


\begin{flushleft}
result
\end{flushleft}


\}





5





\newpage
1.3





\begin{flushleft}
Виконання завдання
\end{flushleft}





\begin{flushleft}
Опишемо головну функцiю для виконання умов завдання та збереження результатiв у виглядi
\end{flushleft}


\begin{flushleft}
таблицi:
\end{flushleft}


\begin{flushleft}
examine $<$- f u n c t i o n ( sample , f i l e n a m e )
\end{flushleft}


\{


\begin{flushleft}
s o r t e d\_sample $<$- s o r t ( sample )
\end{flushleft}


\begin{flushleft}
r e s u l t\_d f $<$- data . frame (
\end{flushleft}


\begin{flushleft}
sample\_mean
\end{flushleft}


\begin{flushleft}
( s o r t e d\_sample )
\end{flushleft}


\begin{flushleft}
g e o m e t r i c\_mean
\end{flushleft}


\begin{flushleft}
( s o r t e d\_sample )
\end{flushleft}


\begin{flushleft}
harmonic\_mean
\end{flushleft}


\begin{flushleft}
( s o r t e d\_sample )
\end{flushleft}


\begin{flushleft}
sample\_median
\end{flushleft}


\begin{flushleft}
( s o r t e d\_sample )
\end{flushleft}


\begin{flushleft}
mid\_r a n g e
\end{flushleft}


\begin{flushleft}
( s o r t e d\_sample )
\end{flushleft}


\begin{flushleft}
variance
\end{flushleft}


\begin{flushleft}
( s o r t e d\_sample )
\end{flushleft}


\begin{flushleft}
s t a n d a r d\_d e v i a t i o n ( s o r t e d\_sample )
\end{flushleft}


\begin{flushleft}
iq
\end{flushleft}


\begin{flushleft}
( s o r t e d\_sample )
\end{flushleft}


\begin{flushleft}
sample\_r a n g e
\end{flushleft}


\begin{flushleft}
( s o r t e d\_sample )
\end{flushleft}


\begin{flushleft}
cv
\end{flushleft}


\begin{flushleft}
( s o r t e d\_sample )
\end{flushleft}


)





,


,


,


,


,


,


,


,


,





\begin{flushleft}
names ( r e s u l t\_d f ) $<$- c (
\end{flushleft}


\begin{flushleft}
{``}mean'' , {``} g e o m e t r i c ␣mean'' , {``} harmonic ␣mean'' ,
\end{flushleft}


\begin{flushleft}
{``} median '' , {``}mid - r a n g e '' , {``} v a r i a n c e '' ,
\end{flushleft}


\begin{flushleft}
{``} s t a n d a r d ␣ d e v i a t i o n '' , {``} i q '' , {``} r a n g e '' , {``} cv ''
\end{flushleft}


)


\begin{flushleft}
write . csv (
\end{flushleft}


\begin{flushleft}
r e s u l t\_df ,
\end{flushleft}


\begin{flushleft}
f i l e =f i l e n a m e
\end{flushleft}


)


\}





\begin{flushleft}
Проведення операцiй з наборами даних в кiнцi програми:
\end{flushleft}


\begin{flushleft}
f i t n e s s\_data $<$- r e a d . c s v (
\end{flushleft}


\begin{flushleft}
f i l e ={``} /home/ f o u r i e r - t r a n s f o r m /R/ r\_p r o j / c o u n t r i e s\_f i t n e s s . c s v '' ,
\end{flushleft}


\begin{flushleft}
h e a d e r=TRUE, s e p={``} ; ''
\end{flushleft}


)


\begin{flushleft}
tomato\_data $<$- r e a d . c s v (
\end{flushleft}


\begin{flushleft}
f i l e ={``} /home/ f o u r i e r - t r a n s f o r m /R/ r\_p r o j / c o u n t r i e s\_tomatoes . c s v '' ,
\end{flushleft}


\begin{flushleft}
h e a d e r=TRUE, s e p={``} ; ''
\end{flushleft}


)


\begin{flushleft}
examine ( f i t n e s s\_data [ , 2 ] , {``} /home/ f o u r i e r - t r a n s f o r m /R/ r\_p r o j / r e s u l t\_frame\_1 . c s v '' )
\end{flushleft}


\begin{flushleft}
examine ( tomato\_data [ , 2 ] , {``} /home/ f o u r i e r - t r a n s f o r m /R/ r\_p r o j / r e s u l t\_frame\_2 . c s v '' )
\end{flushleft}





6





\begin{flushleft}
\newpage
Пiсля виконання вищенаведених iнструкцiй програми отримаємо двi таблицi з певними
\end{flushleft}


\begin{flushleft}
даними.
\end{flushleft}


\begin{flushleft}
result\_frame\_1.csv, в якому зберiгаються результати обчислень пiсля обробки набору даних файлу countries\_fitness.csv:
\end{flushleft}


\begin{flushleft}
mean
\end{flushleft}


456.311428571429


\begin{flushleft}
variance
\end{flushleft}


5626.85281428571





\begin{flushleft}
geometric mean
\end{flushleft}


\begin{flushleft}
harmonic mean
\end{flushleft}


\begin{flushleft}
median mid-range
\end{flushleft}


450.9406123977 445.529580368722 458.16


455.375


\begin{flushleft}
standard deviation
\end{flushleft}


\begin{flushleft}
iq
\end{flushleft}


\begin{flushleft}
range
\end{flushleft}


\begin{flushleft}
cv
\end{flushleft}


75.0123510782439 103.825 208.75 0.164388499567246





\begin{flushleft}
result\_frame\_2.csv, в якому зберiгаються результати обчислень пiсля обробки набору даних файлу countries\_tomatoes.csv:
\end{flushleft}


\begin{flushleft}
mean
\end{flushleft}


33.1042857142857


\begin{flushleft}
variance
\end{flushleft}


32.0136952380952





\begin{flushleft}
geometric mean
\end{flushleft}


\begin{flushleft}
harmonic mean
\end{flushleft}


\begin{flushleft}
median mid-range
\end{flushleft}


32.6618159997933 32.1947846244488


34.1


32.055


\begin{flushleft}
standard deviation
\end{flushleft}


\begin{flushleft}
iq
\end{flushleft}


\begin{flushleft}
range
\end{flushleft}


\begin{flushleft}
cv
\end{flushleft}


5.65806461946974 7.21 15.45 0.170916378269055





\begin{flushleft}
Зауваження: Очевидно, при обчисленнi вибiркової дисперсiї за формулою:
\end{flushleft}


\begin{flushleft}
n
\end{flushleft}





\begin{flushleft}
1X
\end{flushleft}


\begin{flushleft}
S (X) =
\end{flushleft}


\begin{flushleft}
(Xj $-$ X)2 ,
\end{flushleft}


\begin{flushleft}
n j=1
\end{flushleft}


2





\begin{flushleft}
отримали б iншi значення для variance, standard deviation та cv.
\end{flushleft}





7





\newpage
1.4





\begin{flushleft}
Обчислення статистик без використання можливостей комп'ютера
\end{flushleft}





\begin{flushleft}
Виконанi в зошитi..
\end{flushleft}





1.5





\begin{flushleft}
Висновок
\end{flushleft}





\begin{flushleft}
Мова програмування R - це унiверсальний iнструмент для швидких обчислень.
\end{flushleft}


\begin{flushleft}
Якщо мовити про результати аналiзу вищенаведених наборiв даних, то можна дати наступну
\end{flushleft}


\begin{flushleft}
характеристику: розкид першої вибiрки майже еквiвалентний другому. Бiльшiсть значень,
\end{flushleft}


\begin{flushleft}
отриманих за допомогою обчислень без використання комп'ютера, спiвпали з результатами
\end{flushleft}


\begin{flushleft}
роботи програми.
\end{flushleft}





8





\newpage



\end{document}
