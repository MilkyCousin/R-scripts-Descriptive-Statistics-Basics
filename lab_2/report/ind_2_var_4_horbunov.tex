\documentclass{article}

\usepackage[utf8]{inputenc}
\usepackage[T1]{fontenc}
\usepackage[english, ukrainian]{babel}

\usepackage[left=2cm,right=2cm,
top=1cm,bottom=2cm,bindingoffset=0cm]{geometry}
    
\title{
Самостійна робота з дескриптивної статистики
}

\date{6 жовтня 2019 р.}

\author{
Гобунов Даніел Денисович,\\
III курс бакалаврату,\\
комп'ютерна статистика,\\
Варіант №4
}

\begin{document}
\maketitle
\section{Завдання №2}
\subsection{Дані}
Дані, завантажені з сайту компанії Quantnote ( quantquote.com/files/quantquote\_daily\_sp500\_83986.zip ), містять інформацію про котирування на американських фондових біржах акцій компаній, що входять до індексу S \& P 500.
\subsection{Обчислення логарифмів для даних деяких компаній}

Для трьох компаній спробуємо реалізувати методи обчислення наступних значень:

\begin{itemize}
\item Логарифмічні норми прибутку з лагом 1 за змінною clo;
\item Логарифмічні відношення mx/mn;
\item Логарифмічні відношення opn/clo;
\end{itemize}

\subsection{Побудова скриньок. Частина перша}
\subsection{Побудова скриньок. Частина друга}
\subsection{Побудова гістограм. Частина перша}
\subsection{Побудова гістограм. Частина друга}
\subsection{Висновок}
\end{document}